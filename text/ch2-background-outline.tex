% \begin{savequote}[8cm]
% Alles Gescheite ist schon gedacht worden.\\
% Man muss nur versuchen, es noch einmal zu denken.

% All intelligent thoughts have already been thought;\\
% what is necessary is only to try to think them again.
%   \qauthor{--- Johann Wolfgang von Goethe \cite{von_goethe_wilhelm_1829}}
% \end{savequote}

\chapter{\label{ch:2-background}Background}

    \minitoc

    \todo{Introduce that going to introduce notation and give the building blocks this thesis builds off}

    \todo{Nicks comment to revisit: In Sections 2 and 3 you present RL before tree search. This was a surprise to me. Wouldn’t it be better to start with single objective MDPs, then introduce methods of solving them (planning, bandits, rl) and the different assumptions they make? Then move to MO versions. I’d also be open to you presenting MCTS then generalising to MCTS rather than the other way around, but I think the way you present it is probably the most efficient.}


\section{Multi-Armed Bandits}
\label{sec:2-0-mab}

    \todo{Introduce tree search using multi-armed bandits?}
    \hide{
    \begin{itemize}
        \item Would like to think a bit about some of the bandits work that sample actions (from adversarial I think), because they were similar to boltzmann search but I hadn't seen details about those works when writing dents
        \item Also the gradient based MAB stuff in sutton and barto book? Looks relevant? Maybe consider that as update to DENTS paper? Either way, another idea for getting good Go results.
    \end{itemize}
    }
    \todo{list}
    \begin{itemize}
        \item $R(s,a)$ is a random variable in MAB literature, but we're assuming it's a fixed value in RL
        \item Multi-Armed Bandits routines algos
        \item Exploring Bandits routines and algos
        \item Contextual Bandits routines and algos
    \end{itemize}

\section{Markov Decision Processes and Reinforcement Learning}
\label{sec:2-1-rl}

    \todo{list}
    \begin{itemize}
        \item Typical agent interacting with environment diagram 
        \item Agent planning with simulator 
        \item MDPs definition
        \item Value functions (single and multi-objective)
        \item Basic results and definitions we use (tabular planning algorithms)
        \item Talk about entropy and some of that work (probably a subsection)
    \end{itemize}

\section{Trial-Based Heuristic Tree Search and Monte-Carlo Tree Search}
\label{sec:2-2-thts}

    \todo{list}
    \begin{itemize}
        \item Give high level overview of MCTS (why use it etc)
        \item Outline that I'll present this as here is THTS, and then here's the THTS routines for MCTS
    \end{itemize}

    \subsection{Trial Based Heuristic Tree Search}
    \label{sec:2-2-1-thts}
    
        \todo{list}
        \begin{itemize}
            \item Present thts++
            \item Indicate what parts are new versus the original paper (context function, optionally running \mctsmode\ewe and mutli-threading)
            \hide{\item Small comment about multi-threading and two-phase locking used to avoid deadlock}
            \hide{\item \todo{probably not necessary to say - but thought of nice/concise way of explaining it (a node can lock children, not parent, if need info from parent, then it has to put a thread safe copy in the context)}}
            \item Define terms precisely and consistently, for example \mctsmode\ewe (say that notation and terminology varies widely in literature, e.g. does uct run in \mctsmode\ewe or not?)
            \item Mention that $\Vinit$ can be implemented as $V_\theta$ to be used with deep RL methods
            \hide{\item \todo{Find the best place to talk about deep RL? Maybe in the RL section?}}
        \end{itemize}

    \subsection{Monte-Carlo Tree Search}
    \label{sec:2-2-2-mcts}

        \todo{list}
        \begin{itemize}
            \item Give overview of MCTS
            \item Give UCT in terms of THTS schema 
            \item Define terms precisely and consistently in terms of THTS functions, maybe \mctsmode\ewe should go here
            \item Define the value initialisation of THTS using a rollout policy for MCTS
            \item Talk about the things that are ambiguous from literature (e.g. people will just say UCT, which originally presented doesn't run in \mctsmode, but often assumed it does)
            \item Should talk about multi-armed bandits here?
        \end{itemize}
    
    \subsection{Maximum Entropy Tree Search}
    \label{sec:2-2-3-ments}
    
        \todo{list}
        \begin{itemize}
            \item Define MENTS here
        \end{itemize}

\section{Multi-Objective Reinforcement Learning}
\label{sec:2-3-morl}
    
    \todo{list}
    \begin{itemize}
        \item MOMDP definition
        \item (Expected) utility
        \item Define an interface for pareto front and convex hull objects
        \item Define CHVI
        \item Should talk about multi-objective and/or contextual multi-armed bandits here?
        \item I'm planning on aligning this section with the recent MORL survey \cite{morl_survey}
        \item Mention some deep MORL stuff, say that this work (given AlphaZero) is adjacent work
    \end{itemize}

\section{Multi-Objective Monte Carlo Tree Search}
\label{sec:2-4-momcts}

    \todo{I think this whole section can just go in litrev}

    \todo{list}
    \begin{itemize}
        \item Define the old methods (using the CH object methods, so clear that not doing direct arithmetic)
        \item Mention that old method could be written using the arithmetic of CHMCTS (but they don't) 
        \hide{\item TODO: write about \& make sure its implemented - its because just updating for 1 is more efficient in deterministic, and say that the additions can be implemented as updating for 1 value when determinstic}
        \item Different flavours copy UCT action selection, but with different variants
        \item Link back to contributions and front load our results showing that all of the old methods don't explore correctly
    \end{itemize}

\section{Sampling Random Variables}
\label{sec:2-5-sampling}

    \todo{list}
    \begin{itemize}
        \item Talk about the alias method here
        \item Reference to chapter \ref{ch:4-dents} section where talk about using this with THTS
    \end{itemize}
